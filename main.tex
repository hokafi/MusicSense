\documentclass[conference]{IEEEtran}

\usepackage{graphicx}
\usepackage{biblatex}
\usepackage{multicol}
\usepackage[utf8]{inputenc}
\usepackage{xepersian}

\bibliographystyle{ieeetr-fa}
\settextfont{Gandom.ttf}
\addbibresource{bib.bib}
\renewcommand{\thesection}{\arabic{section}}
\usepackage{blindtext, graphicx}
\hyphenation{op-tical net-works semi-conduc-tor}

\usepackage{blindtext, graphicx}

\title{دسته بندی اتوماتیک ژانر موسیقی   }
\author{
امیر حسین باقری\\
مصطفی لسانی\\
حسین کافی
}
\date{پاییز 97}
\renewcommand\footnoterule{\kern-3\hrule\kern2.6}

 

\begin{document}
\maketitle

\section{مقدمه} 
آهنگ ها، یکی از مهم ترین بخش های زندگی روزمره انسان ها می باشد. کمتر پیش می آید که ماها روزانه دقایقی یا ساعتی از عمرمان را صرف گوش دادن به آهنگ ها نکنیم. یکی از مشکلاتی که بسیاری از ماها هنگام گوش دادن به آهنگ ها پیدا میکنیم این است که الان با توجه به حس و حالی که در آن هستیم به چه آهنگی گوش دهیم.
شرکت ها و سازمان های زیادی برروی این مشکل کار کرده اند و به راه حل های جالبی در این زمینه رسیده اند. اکثر آنها با پیاده سازی یک سیستم پیشنهاد دهنده آهنگ کار میکنند که بر دو اساس کار میکند: فیلتر کردن مشارکتی \LTRfootnote{ Collaborative filtering }   و  فیلتر مبنی بر محتوا \LTRfootnote{ Content-based filtering } . در سیستم فیلتر کردن مشارکتی، سیستم در مورد کاربر اطلاعاتی همچون علایق و سلایق را استخراج میکند و با توجه به اینکه کاربر های مشابه به چه آهنگ هایی گوش میدهند این آهنگ ها را به کاربر پیشنهاد میدهد. در مکانیزم فیلتر مبنی بر محتوا ، با استفاده از اطلاعات مربوط به آهنگ ها همچون ژانر، سال ، خواننده و ... آهنگ های مشابه را پیدا کرده و به کاربران نشان میدهند. در هر دو مکانیزم اطلاعات زیادی نیاز است تا این سیستم به درستی کار کند. در حالت فیلتر مشارکتی به تعداد زیادی کاربر نیاز است که علایق آنها مشخص شده و در حالت فیلتر مبنی بر محتوا نیز اطلاعات زیادی هم چون ژانر آهنگ ، سالی که آهنگ انتشار یافت ، خواننده و ... نیاز است. کاری که ما انجام دادیم بررسی سیگنال های خام آهنگ بوده و بر اساس آن و استفاده از الگوریتم های خوشه بندی ،آهنگ های همانند هم در یک خوشه قرار میگیرند. 
ژانر ، برچسبی است که برای دسته بندی آهنگ های مشابه به کار میرود. اگر چه ژانر معمولا به صورت دستی توسط انسان ها مشخص می شود ، ما سعی داریم ژانر آهنگ ها را از طریق سیگنال خام آهنگ ها تخمین بزنیم. در این مقاله ما سیگنال آهنگ ها را بررسی کردیم و ویژگی های خاصی که در آهنگ وجود دارد را استخراج کردیم. سپس با استفاده از تکنیک های یادگیری ماشین آهنگ های مشابه را دسته بندی کردیم. سپس بررسی کردیم که چه مقدار این روش برای تشخیص درست ژانر آهنگ میتواند موثر باشد.


\end{document}
